\chapter*{ВВЕДЕНИЕ}

\addcontentsline{toc}{chapter}{ВВЕДЕНИЕ}

Последнее десятилетие в нашей стране отмечается повышенным вниманием к геоинформатике.
За это время опубликовано большое количество статей и книг, посвященных вопросам и проблемам связанным с ней.
Вместе с тем сравнительно недавно возникла потребность работы с географическими данными через среду Internet.
Это связано с рядом факторов: широким распространением Internet--технологии в Российской Федерации, появлением высокопропускных каналов, развитием микроэлектронной базы ЭВМ~\cite{intro-1}.

Пространственные данные необходимы для предупреждения природных явлений различного рода (например, анализ сейсмической активности при выборе этажности планируемой застройки), в процедурах городского планирования (выбор места постройки больницы, детского сада и аналогичных объектов социальной инфраструктуры).
Геоинформационные системы и технологии являются мощным инструментом информационной поддержки процессов планирования и управления в транспортных системах~\cite{transport-geo}.

Целью работы является классификация методов поиска пространственных данных в реляционных базах данных.

Чтобы достигнуть поставленной цели, необходимо решить следующие задачи:
\begin{itemize}
    \item провести анализ предметной области географических (пространственных) данных;
    \item провести обзор существующих методов хранения географических (пространственных) данных в реляционных СУБД;
    \item сформулировать критерии классификации методов поиска географических (пространственных) данных;
    \item классифицировать методы поиска географических (пространственных) данных.
\end{itemize}
