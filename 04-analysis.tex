\chapter{Аналитический раздел}

Необходимо рассмотреть способы организации хранения пространственных данных в реляционных СУБД.

\section{Способы хранения географических данных в реляционных СУБД}

\subsection{Использование XML}

% TODO

\subsection{Использование JSON}

Для хранения географических (пространственных) данных возможно использовать JSON в случае, если СУБД поддерживает этот тип данных.

GeoJSON это основанный на JSON открытый стандарт и формат, созданный для представления простых географических объектов наряду с их непространственными атрибутами~\cite{geojson}.

% TODO

\subsection{Использование расширений СУБД}

В настоящее время многие известные системы управления реляционными базами данных предлагают возможности хранения географических (пространственных) данных с применением сторонних расширений~\cite{rdbms-spatial-extensions}.

Для СУБД Postgres~\cite{pg-docs} существует расширение PostGIS~\cite{postgis-docs}, а для Oracle~\cite{oracle-docs} --- Oracle Spatial~\cite{oracle-spatial-docs}.

Расширения СУБД для работы с географическими (пространственными) данными вводят новые типы данных для хранения геометрий (точка, линия, полигон), а также операции работы с ними.

Преимущества:
\begin{enumerate}
    \item 
\end{enumerate}

Недостатки:
\begin{enumerate}
    \item Необходимо детально разрабатывать схему хранения данных~\cite{rdbms-spatial-extensions}.
    \item Трудоемкий процесс загрузки данных в связи с возможной необходимостью преобразований аттрибутов и геометрий~\cite{rdbms-spatial-extensions}.
\end{enumerate}

\section{Методы поиска}

\subsection{Последовательный поиск}

Последовательный поиск -- это метод доступа к данным, при котором СУБД читает все строки таблицы, чтобы найти те, которые соответствуют заданным критериям.
Этот метод может быть эффективен в следующих случаях:
\begin{itemize}
    \item отсутствие индексов;
    \item большая выборка данных в результате выполнения запроса, вернувшего значительное число строк (более 50\% от общего числа). Связано с тем, что в данном случае последовательный поиск более эффективен в связи с минимизацией количества операций ввода--вывода~\cite{sequence-scan}.
\end{itemize}

Последовательный поиск стоит применять в следующих ситуациях:
\begin{itemize}
    \item небольшие таблицы: для небольших наборов данных последовательный поиск может быть быстрым и эффективным;
    \item запросы на большую выборку: если запрос возвращает много строк или фильтрация по критериям не очень узкая (например, выбираются все записи за определённый период), то последовательный поиск будет предпочтительным~\cite{sequence-scan}.
\end{itemize}

\subsection{Использование индекса Quadtree}

Индекс Quadtree вычисляет аппроксимации фрагментов для геометрий и использует существующие индексы B--дерева для выполнения пространственного поиска и других операций с DML~\cite{quadtree}.

QuadTree — это иерархическая структура данных, используемая для разбиения двумерного пространства на четыре квадранта или подрегионы.
Каждый узел дерева представляет собой квадрат, который может быть разделён на четыре равных квадрата, если в нём содержится больше определённого количества объектов.
Это позволяет эффективно выполнять операции поиска, такие как нахождение всех объектов в заданной области~\cite{shekhar2003spatial}.

Дерево квадрантов работает эффективно для больших запросов на разреженном наборе данных.
А в случае неравномерно распределенных данных его эффективность снижается~\cite{unevenly-distributed}.

На рисунке \ref{img:quadtree} изображена визуализация структуры дерева квадрантов (quadtree).

\includeimage
{quadtree}
{f}
{H}
{1\textwidth}
{Визуализация структуры Quadtree}

\subsection{Использование индекса R--Tree}

Индекс R--Tree (regions tree) реализуется логически в виде дерева и физически с использованием таблиц внутри базы данных.

Дерево регионов представляет собой иерархическую структуру, где каждый узел может содержать переменное количество элементов, которые представляют собой ограничивающие прямоугольники (bounding boxes) для групп объектов.

На рисунке \ref{img:rtree} изображена визуализация структуры R--Tree.

\includeimage
{rtree}
{f}
{H}
{1\textwidth}
{Визуализация структуры R--Tree}

Процедура вставки нового элемента в R--дерево состоит из:
\begin{enumerate}
    \item Поиск подходящего узла: находится узел, в который можно добавить элемент, минимизируя увеличение объема его ограничивающего прямоугольника.
    \item Добавление элемента: если узел переполняется (превышает максимальное количество элементов), он разделяется на два узла с использованием алгоритмов кластеризации, таких как квадратичное или линейное разбиение.
    \item Поддержание баланса: структура R--дерева остается сбалансированной, что обеспечивает эффективный поиск.
\end{enumerate}

Квадратичное разбиение заключается в разбиении на два прямоугольника с минимальной площадью, покрывающие все объекты.
Линейный – в разбиении по максимальной удаленности.

Поиск в R--дереве осуществляется путем рекурсивного обхода дерева, начиная с корня.
Каждый узел содержит информацию о своих дочерних узлах и ограничивающих прямоугольниках, что позволяет быстро отсеивать ненужные ветви дерева.

\section{Сравнительная таблица индексов, основанных на деревьях}

\begin{table}[H]
    \begin{tabular}{|l|l|l|}
        \hline
        Характеристика                                                            & R-Tree                                                                                                   & Quadtree                                                                                             \\ \hline
        \textbf{Структура}                                                        & \begin{tabular}[c]{@{}l@{}}Иерархическая структура \\ с ограничивающими \\ прямоугольниками\end{tabular} & \begin{tabular}[c]{@{}l@{}}Делит пространство на \\ квадраты\end{tabular}                            \\ \hline
        \textbf{Балансировка}                                                     & Сбалансированное дерево                                                                                  & \begin{tabular}[c]{@{}l@{}}Не обязательно сбалан-\\ сировано\end{tabular}                            \\ \hline
        \textbf{Перекрытие узлов}                                                 & Узлы могут перекрываться                                                                                 & Узлы не перекрываются                                                                                \\ \hline
        \textbf{\begin{tabular}[c]{@{}l@{}}Наличие пустых\\ листьев\end{tabular}} & Невозможно                                                                                               & Возможно                                                                                             \\ \hline
        \textbf{Обновления}                                                       & \begin{tabular}[c]{@{}l@{}}Могут требовать пере-\\ стройки узлов\end{tabular}                            & \begin{tabular}[c]{@{}l@{}}Легче обновлять без \\ значительных изменений \\ в структуре\end{tabular} \\ \hline
    \end{tabular}
\end{table}

\section*{Вывод}
